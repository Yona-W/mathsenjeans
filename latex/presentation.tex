\documentclass[french]{beamer}
\usepackage[frenchb]{babel}
\usepackage{etex}
\usepackage[T1]{fontenc}
\usepackage[latin1]{inputenc}
\usepackage{pst-all}
\usepackage{graphicx}
\usepackage{fourier}
\usetheme{Warsaw}
\title{Le jeu de l'Awal�}
\date{Avril 2016}
\author{SCHICHL Arthur, KAVAN Lukas, MONTINERI Jonathan \\ Lyc�e Francais de Vienne}

\begin{document}


\begin{frame}
\titlepage
\end{frame}

\section{Le jeu}

\begin{frame}
\frametitle{Les r�gles}
\begin{itemize}
\onslide<2->\item Deux parties � six cases, $4$ pierres dans chaque au d�but.
\onslide<3->\item Si la derni�re pierre tombe sur une case de $1$ ou $2$ pierre(s), il capture les pierres.
\onslide<4->\item Si le joueur ne peut plus jouer, l'adversaire obtient les pierres restantes.
\onslide<5->\item Celui que a les plus de pierres a gagn�.
\end{itemize}
\end{frame}


\section{Strat�gies}

\begin{frame}
\frametitle{Strat�gies}
\framesubtitle{...de d�but}
\begin{itemize}
\onslide<2->\item �viter des tas de $2$ et de $1$.
\onslide<3->\item Construire un grand tas.
\onslide<4->\item Garder les pierres le plus longtemps possible.
\end{itemize}
\end{frame}

\begin{frame}
\frametitle{Strat�gies}
\framesubtitle{...de milieu}
\begin{itemize}
\onslide<2->\item Les grands tas semblent �tre avantageux.
\onslide<3->\item Pr�parer le terrain de jeu de l'adversaire.
\onslide<4->\item Faire en sorte que l'adversaire n'ait aucun autre choix que de s'�liminer soi-m�me.
\end{itemize}
\end{frame}

\begin{frame}
\frametitle{Strat�gies}
\framesubtitle{...de fin}
\begin{itemize}
\onslide<2->\item Prolonger la partie et gagner du temps.
\onslide<3->\item D�placer les pierres une par une.
\onslide<4->\item Ne pas donner de pierres � l'adversaire.
\end{itemize}
\end{frame}


\section{Notre programme}
\subsection{Principe}
\begin{frame}
\frametitle{Explication}
\begin{itemize}
\onslide<2-5>\item Le programme cherche les meilleurs coups.
\onslide<3-5>\item � chaque coup possible est associ� une valeur.
\onslide<4-5>\item Ensuite le coup avec la meilleure valeur est jou�.
\end{itemize}
\onslide<5->\center{Comment cette valeur est-elle calcul�e?}
\end{frame}

\subsection{L'Intelligence}

\begin{frame}
\frametitle{Nos crit�res d'impl�mentation}
\framesubtitle{L'avancement du jeu}
\begin{itemize}
\onslide<2->\item D�termin� en grande partie par le nombre de boules en jeu.
\onslide<3->\item La pr�sence d'un grand tas (sup�rieur � 12).
\onslide<4->\item Permet la diff�renciation entre les types de strat�gies.
\end{itemize}
\end{frame}

\begin{frame}
\frametitle{Nos crit�res d'impl�mentation}
\framesubtitle{L'�tat du terrain}
\begin{itemize}
\onslide<2->\item "Fonctions d'�valuation".
\onslide<3->\item But: D�terminer la "qualit�" d'un �tat du terrain.
\onslide<4->\item Une fonction par crit�re (positif ou n�gatif).
\end{itemize}
\end{frame}

\subsection{L'algorithme r�sum�}

\begin{frame}
\frametitle{Le fonctionnement}
\onslide<2-> Pour chaque coup possible:  
\onslide<3-> \\Essayer de jouer ce coup
\onslide<4-> \\�valuer son efficacit�
\onslide<5-> \\Ensuite, pour le meilleur des coups:
\onslide<6-> \\Si ce coup est l�gal, jouer ce coup puis attendre le joueur.
\end{frame}


\begin{frame}
\frametitle{Joignez nous!}
\center{\Huge{Connectez-vous � notre r�seau WiFi et jouez � Awal�}}
\end{frame}


\end{document}
